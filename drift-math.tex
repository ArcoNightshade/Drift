\documentclass{article}
\usepackage{amsmath}
\usepackage{amssymb}
\usepackage{geometry}
\geometry{a4paper, margin=1in}

% --- Fancier Formatting Packages ---
\usepackage{xcolor}
% Define colors based on your palette (Deep Void and a Slate Gray)
\definecolor{DeepVoid}{HTML}{2A3B4A}
\definecolor{SlateGray}{HTML}{7C808B}
\usepackage[most]{tcolorbox}
\usepackage{titlesec}
\usepackage{wasysym} % For special symbols like 'sun'

% Customize Section Headings
\titleformat{\section}[block]{\centering\huge\bfseries\color{DeepVoid}\tcbline}{}{1em}{}
\titleformat{\subsection}[runin]{\Large\bfseries\color{SlateGray}}{}{0pt}{}
\titleformat{\subsubsection}[hang]{\bfseries\color{DeepVoid}}{}{1em}{}
\titlespacing*{\section}{0pt}{2ex}{1ex}
\titlespacing*{\subsection}{0pt}{1.5ex}{1ex}
\titlespacing*{\subsubsection}{0pt}{1ex}{0.5ex}

\tcbuselibrary{skins}

% Define a custom environment for equations to give them a colored background
\tcbset{
    enhanced,
    colback=DeepVoid!5!white, % Light colored background
    colframe=DeepVoid!60!black, % Dark colored frame
    boxrule=0.5pt,
    arc=3pt,
    left=5pt,
    right=5pt,
    top=5pt,
    bottom=5pt,
}

\newenvironment{mathbox}[1][\unskip]{
    \begin{tcolorbox}
    \textbf{#1}
}{
    \end{tcolorbox}
}
% --- End Fancier Formatting Packages ---

\title{\color{DeepVoid}\textbf{Hollow Orbit: Advanced Survival Mechanics}}
\author{\large Mathematical Core Documentation}
\date{Version 2.0}

\begin{document}
\maketitle
\thispagestyle{empty} % Suppresses page number on the first page

\section*{Orbital Decay and System Mechanics}

The core survival loop is modeled using deterministic equations that quantify the inevitable collapse of the habitat's orbit, compounded by component failure and resource scarcity. The model is designed to create a sense of mounting dread and irreversible decline.

\subsection{The Core Decay Metric: Orbital Integrity ($I$)}\hspace{0.5em}

Orbital Integrity ($I$) represents the habitat's overall stability. The rate of loss ($\frac{dI}{dt}$) is a function of base decay, environmental drag, system instability, and crucially, power sufficiency.

\begin{mathbox}[\color{DeepVoid}Orbital Integrity Loss Rate]
$$
\frac{dI}{dt} = - (R_{\text{Base}} + R_{\text{Drag}} + R_{\text{Instability}}) \cdot \Phi_{\text{Power}}
$$
\end{mathbox}

\begin{itemize}
    \item $I$: \textbf{Orbital Integrity} (0--100). When $I \le 0$, the orbit is lost.
    \item $R_{\text{Base}}$: \textbf{Inherent Decay Rate}. The constant, unavoidable rate of decline specific to the current orbit.
    \item $\Phi_{\text{Power}}$: \textbf{Power Sufficiency Multiplier}. If global available Power ($P$) drops below a critical threshold ($P_{\text{Crit}}$), essential gyroscopes and sensors fail, forcing $\Phi_{\text{Power}} > 1$ and exponentially accelerating $I$ decay.
\end{itemize}

\vspace{1em}
\subsection{Compounding Decay Rates}\hspace{0.5em}

\subsubsection{Atmospheric Drag ($R_{\text{Drag}}$)}
Drag is a dynamic threat, increasing as the habitat's \textbf{Orbital Altitude ($A$)} drops, which is itself poorly maintained by a damaged Hull System ($H_{\text{System}}$).

\begin{mathbox}[\color{DeepVoid}Atmospheric Drag Penalty]
$$
R_{\text{Drag}} = \alpha \cdot \left(1 - \frac{H_{\text{System}}}{100}\right)^2 \cdot \frac{1}{\min(1, A/A_{\text{Threshold}})}
$$
\end{mathbox}

\begin{itemize}
    \item $A$: \textbf{Current Orbital Altitude}. Altitude degrades independently if thruster maintenance is neglected.
    \item $H_{\text{System}}$: Hull and Maintenance System Integrity (0--100).
    \item The $\frac{1}{\min(\dots)}$ factor ensures that if $A$ drops close to the $A_{\text{Threshold}}$ (atmospheric boundary), the drag penalty spikes dramatically, creating a rapid descent failure state.
\end{itemize}

\subsubsection{System Instability ($R_{\text{Instability}}$)}
Instability is the accumulated penalty from the current failure state of critical internal components (e.g., CPU, Gyroscope).

\begin{mathbox}[\color{DeepVoid}Total Instability Rate]
$$
R_{\text{Instability}} = \sum_{n=1}^N W_n \cdot B_n
$$
\end{mathbox}

\begin{itemize}
    \item $W_n$: \textbf{Component Weight/Criticality}. A higher weight indicates a more vital component (e.g., $W_{\text{Gyro}} > W_{\text{Lights}}$).
    \item $B_n$: \textbf{Broken State Multiplier}. $B_n=1$ if Component Integrity $\le 0$; $B_n=0$ otherwise.
    \item \textbf{Probabilistic Failure}: Component failure is not always deterministic. At every tick, each component has a small chance ($\lambda_n$) of suffering a catastrophic failure (Integrity $\to 0$).
    $$
    \lambda_n = C_{\text{Failure}} \cdot \frac{R_{\text{Base}} \cdot \text{Age}_n}{\text{Max } H_{n}}
    $$
    This models the effect of age ($\text{Age}_n$) and the inherently high stress of the current orbit ($R_{\text{Base}}$) on fragile hardware.
\end{itemize}

\vspace{1em}
\subsection{Player Actions and Resource Management}\hspace{0.5em}

\subsubsection{Repair Action (Component Integrity Gain)}
Repair efficiency is penalized by the severity of the current orbit, meaning resources go less far in later stages of the game.

\begin{mathbox}[\color{DeepVoid}Repair Efficiency]
$$
\Delta H_{\text{System}} = \frac{\beta \cdot P_{\text{Consumed}} \cdot R_{\text{es}_{\text{Consumed}}}}{\text{Max } H_{\text{System}}} \cdot \frac{1}{1 + \delta \cdot R_{\text{Base}}}
$$
\end{mathbox}

\begin{itemize}
    \item $\beta$: \textbf{Repair Efficiency Constant}. A baseline measure of tool effectiveness.
    \item $\frac{1}{1 + \delta \cdot R_{\text{Base}}}$: \textbf{Difficulty Multiplier}. As the base decay rate increases ($\delta \cdot R_{\text{Base}}$ grows), the repair yield drops, ensuring the player faces an ever-increasing maintenance burden.
\end{itemize}

\subsubsection{Orbital Correction (Emergency Burn)}
The high-cost, high-risk maneuver to temporarily regain altitude by consuming Fuel ($F$). Its effectiveness is dependent on the health of the propulsion system.

\begin{mathbox}[\color{DeepVoid}Altitude Correction Gain]
$$
\Delta A = \eta \cdot F_{\text{Consumed}} \cdot \min\left(1, \frac{H_{\text{Thruster}}}{100}\right)
$$
\end{mathbox}

\begin{itemize}
    \item $\eta$: \textbf{Fuel Efficiency Constant}.
    \item $H_{\text{Thruster}}$: Integrity of the specific thruster component. A damaged thruster results in wasted fuel and lower $\Delta A$, adding uncertainty to emergency burns.
\end{itemize}

\vspace{1em}
\subsection{Orbit Transfer and Game Progression}\hspace{0.5em}

Upon transferring to a new habitat, the game progresses to an inherently more challenging state, ensuring the player is constantly fighting an uphill battle.

\begin{mathbox}[\color{DeepVoid}Next Orbit Difficulty Scaling]
$$
\hat{R}_{\text{Base}} = R_{\text{Base}_{\text{Previous}}} + \gamma
$$
$$
\hat{A}_{\text{Initial}} = A_{\text{Initial}_{\text{Previous}}} - \kappa
$$
\end{mathbox}

\begin{itemize}
    \item $\hat{R}_{\text{Base}}$: The new, higher inherent decay rate ($\gamma > 0$ is the constant increase).
    \item $\hat{A}_{\text{Initial}}$: The starting altitude of the new orbit is lower ($\kappa > 0$ is the constant reduction), guaranteeing a severe initial drag challenge.
    \item This mechanism reinforces the core narrative: each sanctuary is merely a temporary reprieve before a worse challenge.
\end{itemize}

\end{document}
